\section{DDALAB}

{\DDALAB} is a cross-platform desktop application for Delay Differential Analysis of EEG data, featuring a hybrid architecture that supports both standalone embedded processing and institutional server deployments. Built on the Tauri framework, it combines web technologies with native Rust performance while providing zero-configuration deployment options.

\paragraph{Architecture and Deployment Modes}
The platform offers two deployment strategies: (1) a standalone desktop application with an embedded Rust API requiring no external dependencies, and (2) a containerized FastAPI server for institutional deployments using Docker orchestration with PostgreSQL, Redis, MinIO storage, and Traefik SSL management. Both modes utilize a custom EDF reader implementation ensuring accurate file parsing and duration calculations. The embedded API processes all data locally for complete privacy, while the FastAPI mode enables collaborative workflows with centralized storage. Authentication via JWT tokens and native state persistence operate seamlessly across both deployment modes.

\paragraph{EEG Processing and Analysis Pipeline}
The platform processes EEG files through a robust pipeline that validates headers, extracts metadata (sampling rates, channel counts, duration), and performs intelligent channel selection. The Docker-based server leverages the \verb|dda_py| Python package~\cite{dda-py}, enhanced with APE (Actually Portable Executable) compatibility for cross-platform execution. The embedded Rust API executes DDA binaries directly through native process management. Both implementations support multiple algorithm variants (ST, CT, CD, DE) with comprehensive error handling, including panic recovery for corrupted files in the embedded API. The \verb|dda_py| package is published at \url{https://pypi.org/project/dda-py/}.

\paragraph{Visualization and User Interface}
Sophisticated visualizations combine web-based plotting with optimized native rendering, supporting heatmaps with log-transformed color mapping (Viridis, Plasma, Inferno, Jet, Cool, Hot) and detailed line plots with scaling exponent annotations. The system automatically loads recent analyses on startup, handles large datasets efficiently through sample-based and time-based coordinate conversions, and supports pop-out windows with export capabilities for publication-quality figures. Comprehensive preprocessing is available including frequency filtering, notch filtering, detrending, resampling, outlier removal, normalization, and smoothing.

\paragraph{State Management and Performance}
The Tauri architecture enables sophisticated state persistence across sessions, preserving workspace configurations, analysis results, visualization settings, selected backend mode, and data directory preferences. The embedded API stores results locally while FastAPI deployments synchronize with MinIO for collaborative access. For institutional deployments, Prometheus and Grafana provide comprehensive monitoring of execution times, memory usage, and error rates. The embedded API uses Rust's structured logging for performance tracking without external infrastructure.

\paragraph{Security and Error Handling}
The embedded API ensures complete data privacy by processing all files locally on the user's device with no external transmission. Institutional deployments maintain data within controlled networks using containerized service isolation and SSL encryption. Multi-layered error handling includes panic recovery, graceful degradation for malformed files, structured error responses, and automatic state recovery. Native file system permissions and memory-safe Rust implementation provide robust security.

\paragraph{Updates and Distribution}
Automatic updates are delivered via Tauri's Updater component for standalone installations and Docker's update mechanisms for server deployments, with API back-compatibility ensuring older clients work with newer servers. GitHub Actions automatically builds and publishes releases at \url{https://github.com/sdraeger/DDALAB}. A workshop held in August provided training on {\DDALAB} and DDA algorithms, with recordings available at \url{https://snl.salk.edu/~claudia/DDALAB/workshop.html}. Additional information is available at \url{https://snl.salk.edu/~claudia/DDALAB/ddalab.html}.

\paragraph{Future Development}
Planned enhancements include integration with Brain Imaging Data Structure~\cite{gorgolewski2016brain}, OpenNeuro~\cite{markiewicz2021openneuro}, and NEMAR~\cite{delorme2022nemar} interfaces. Development will expand preprocessing capabilities in the embedded API to match FastAPI features and implement a MATLAB-inspired session capability for exporting analysis workflows to Julia and Python code.
